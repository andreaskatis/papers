\section{Introduction}
The problem of automated synthesis of reactive systems using from propositional specifications is a very well studied area of research~\cite{gulwani2010dimensions}. By definition, the problem of synthesis entails the discovery of efficient algorithms able to construct a candidate program that is guaranteed to comply with the predefined specification. Inevitably, the related work on synthesis has tackled several sub-problems, such as that of function and template synthesis, as well as the weaker problem regarding the implementability, or otherwise, realizability of the specification.

In a similar fashion, a collaboration between Rockwell Collins and
the University of Minnesota has focused on designing tools that provide
compositional proofs of correctness~\cite{NFM2012:CoGaMiWhLaLu,Whalen13:WhatHow:TwinPeaksIEEESoftware,hilt2013,QFCS15:backes}.
In the context of synthesis, we recently introduced a decision procedure for determine the {\em realizability} of contracts involving infinite theories such as the linear integer/real arithmetic and/or uninterpreted functions that is checkable by any SMT solver that supports quantification~\cite{Katis15:Realizability}. Furthermore, in~\cite{Katis:machine} we formally proved the soundness of our checking algorithm using the Coq interactive theorem prover. The realizability checking procedure is now part of the AGREE reasoning framework~\cite{NFM2012:CoGaMiWhLaLu}, which supports compositional
assume/guarantee contract reasoning over system architectural models written in
AADL~\cite{SAE:AADL}.

While checking the realizability of contracts provided us with fruitful results
and insight in several case studies, it also worked as solid ground towards the
development of an automatic component synthesis procedure. The
most important obstacle initially, was the inability of the SMT-solver to
handle higher-order quantification. Fortunately, interesting directions to
solving this problem have already surfaced, either by extending an SMT-solver
with native synthesis capabilities\cite{reynoldscounterexample}, or by providing
external algorithms that reduce the problem by efficient quantifier elimination methods~\cite{fedyukovichae}.

The main contribution of this paper is, therefore, the implementation of a
component synthesis algorithm for infinite theories, using specifications
expressed in assume-guarantee reasoning.  The algorithm heavily relies on our previous implementation for realizability checking, but also takes advantage of a recently published skolemizer for $\forall\exists$-formulas, named
AE-VAL. The main idea in this implementation is to effectively extract a Skolem
relation that is essentially, a collection of strategies, that can directly lead to an implementation which is guaranteed to comply to the corresponding contract.

In Section~\ref{sec:preliminaries} we provide the necessary background
definitions from our previous work on realizability checking.
Section~\ref{sec:synthesis} presents our approach to solving the synthesis
problem for assume-guarantee contracts using theories. Finally, in
Section~\ref{sec:related work} we give a brief historical background on the
related research work on synthesis, and we report our conclusions and upcoming
future work in Section~\ref{sec:conclusion}.