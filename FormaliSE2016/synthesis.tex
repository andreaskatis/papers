\section{Synthesis from Contracts}
\label{sec:synthesis}

With a sound implementation of the realizability checking algorithm at
hand, the next step was to tackle the more
interesting problem of synthesis, i.e. the automated derivation of
implementations that would be safe in terms of satisfying the constraints
defined by the component's contract. The intuition behind solving the
synthesis problem in our context relies on finding a set of initial states
$I$ and a transition relation $T$ that would satisfy the requirements
specified in the contract. Unfortunately, the lack of power in SMT solvers in
terms of solving formulas that contain higher-order quantification immediately
ruled out the prospect of using one as our primary synthesis tool. Therefore, an alternative work from Fedyukovich et al.~\cite{fedyukovichae,fedyukovich2014automated} on a skolemizer for $\forall\exists$-formulas on linear integer arithmetic was chosen to be used as a means of extracting a witness that could directly be used in component synthesis.

The tool, called AE-VAL is using the Model-Based Projection technique
in~\cite{komuravelli2014smt} to validate $\forall\exists$-formulas, based on
Loos-Weispfenning quantifier elimination~\cite{loos1993applying}.
As part of the procedure, a Skolem relation is provided for the existentially quantified variables of the formula. The algorithm
initially distributes the models of the original formula into disjoint
uninterpreted partitions, with a local Skolem relation being computed for each
partition in the process. From there, the use of a Horn-solver provides an
interpretation for each partition, and a final global Skolem relation is produced.

The idea behind our approach to solving the synthesis problem is simple.
Consider the checks~\ref{eq:sbcheck} and~\ref{eq:echeck} that the realizability
checking algorithm is using. \textit{BaseCheck'} is still
necessary for the synthesis problem to ensure that all initial states in the problem are valid.
\textit{ExtendCheck} on the other hand can be further used in actually
synthesizing implementations. The check tries to prove that every valid state in
our system is extendable, i.e. all states can be starting points to paths that
comply to the system contract, and furthermore are extendable by one step:
	\begin{multline*}
		\forall i_1,s_1,\ldots,i_n,s_n. \\
		\hspace{-2cm} A(s,i_1) \wedge G_T(s,i_1,s_1) \wedge \ldots \wedge \\
		A(s_{n-1}, i_n) \wedge G_T(s_{n-1},i_n,s_n) \Rightarrow \\
		\hspace{+2cm} \forall i. A(s_n,i) \Rightarrow \exists s'. G_T(s_n,i,s')
	\end{multline*}

which can be rewritten:

	\begin{multline}
	\label{ml:extendable2}
		\forall i_1,s_1,\ldots,i_n,s_n,i. \\
		\hspace{-2cm} A(s,i_1) \wedge G_T(s,i_1,s_1) \wedge \ldots \wedge \\
		A(s_{n-1}, i_n) \wedge G_T(s_{n-1},i_n,s_n) \wedge A(s_n,i) \Rightarrow \\
		\hspace{+2cm} \exists s'. G_T(s_n,i,s')
	\end{multline}



\noindent Such a formula is exactly what is required by a $\forall\exists$
solver such as AE-VAL in order to produce the witness for the existential variables $s'$.
%
AE-VAL solves this formula by providing an assignment for each existential variable in a piecewise relation based on a partitioning based on assignments to the universal variables.  In other words, by examining a bounded history of the state and input variable values in the contract (the universally quantified variables in Formula~\ref{ml:extendable2}), we determine the next values of the state variables.  An example of a portion of this partitioning is shown in Figure~\ref{fig:skolem-rel}.
In other words, the Skolem relation contains, starting from a valid
initial state of variables, strategies in terms of how the new state is
selected, in such a way, that the contract is not violated.
\iffalse
This is particularly
interesting in our case since in the underlying machinery, each contract is
translated into an equivalent program in the Lustre specification language,
which explicitly defines expressions that dictate the way in which each variable
is expected to behave after a transition occurs to a new state. Therefore, a
final implementation is a simple Lustre program that--apart from the initial
state definitions--contains a set of expressions that depict the behaviour of
each variable under a certain condition, in an equivalent manner to that
indicated by the Skolem relation.
\fi

\begin{figure}
\label{fig:algorithm}
\begin{small}
\begin{verbatim}
// for each variable in I or S,
//   create an array of size k.
//   then initialize initial state values
assign_GI_witness_to_S;
update_array_history;

// Perform bounded 'base check' synthesis
read_inputs;
base_check'_1_solution;
update_array_history;
...
read_inputs;
base_check'_k_solution;
update_array_history;

// Perform recurrence from 'extends' check
while(1) {
 read_inputs;
 extend_check_k_solution;
 update_array_history;
}
\end{verbatim}
\end{small}
\caption{Algorithm skeleton for synthesis}
\end{figure}

\noindent

Thus, we can construct the skeleton of an algorithm as shown in Figure~\ref{fig:algorithm}.  
We begin by creating an array for each input and history variable up to depth
$k$, where $k$ is the depth at which we found a solution to our realizability algorithm.
In each array, the zero'th element is the `current' value of the variable, the first element is the previous value, and the $k-1$'th value is the $k-1$-step previous value.
We then generate witnesses for each of the {\em BaseCheck'} instances of
successive depth using the AE-VAL solver to describe the initial behavior of the
implementation up to depth $k$.  This process starts from the memory-free
description of the initial state ($G_I$).  There are two `helper' operations:
{\em update\_array\_history} shifts each array's elements one potition forward
(the $k-1$'th value is simply forgotten), and {\em read\_inputs} reads the current values of inputs into the zeroth element of the input variable arrays.  Once the history is entirely initialized using the {\em BaseCheck'} witness values, we enter a recurrence loop where we use the solution of the {\em ExtendCheck} to describe the next value of outputs.
 
\iffalse
Taking this query and providing it as a target formula for AE-VAL effectively
gives an answer to the synthesis problem in the form of a Skolem relation for
the candidate states that can further extend valid execution paths in the
system. In other words, the Skolem relation contains, starting from a valid
initial state of variables, strategies in terms of how the new state is
selected, in such a way, that the contract is not violated. This is particularly
interesting in our case since in the underlying machinery, each contract is
translated into an equivalent program in the Lustre specification language,
which explicitly defines expressions that dictate the way in which each variable
is expected to behave after a transition occurs to a new state. Therefore, a
final implementation is a simple Lustre program that--apart from the initial
state definitions--contains a set of expressions that depict the behaviour of
each variable under a certain condition, in an equivalent manner to that
indicated by the Skolem relation.
\fi

\subsection{Synthesis Example}
As an example that demonstrates the process, consider the contract created for
a mode controller in a simple microwave model of 260 lines of code that was
used as one of the base case studies in~\cite{Katis15:Realizability}. The
controller has four inputs, \textit{start} which is used to indicate whether the microwave is at an initial
state or not, \textit{clear} that is used as a stop signal for the system,
\textit{seconds\_to\_cook} as a countdown timer and \textit{door\_closed} as an
indicator that the microwave's door is closed or not. The controller returns the
current state of the microwave's mode using \textit{cooking\_mode}. The contract
consists of one assumption and nine guarantees, which are shown below
informally, as well as formally in AADL. A library named \textit{defs} is used
to define auxiliary functions, such as \textit{rising\_edge()} which returns ``true'' when the corresponding signal is at its rising edge, and \textit{initially\_true()} which is used to
check a variable's value at the component's initial state.

\begin{requirement}
\textbf{MC Assumption} - seconds\_to\_cook is greater than or equal to
zero.
\begin{verbatim}
seconds_to_cook >= 0;
\end{verbatim}
\label{req:micasm}
\end{requirement}
%
\begin{requirement}
\textbf{MC Guarantee-0} - The range of the cooking\_mode variable shall be
[1..3].
\begin{verbatim}
cooking_mode >= 1 and cooking_mode <= 3 ;
\end{verbatim}
\label{req:mic0}
\end{requirement}
%
\begin{requirement}
\textbf{MC Guarantee-1} - The microwave shall be in cooking mode only when the
door is closed.
\begin{Verbatim}[obeytabs,fontsize=\small]
is_running => door_closed;
\end{Verbatim}
\label{req:mic1}
\end{requirement}

\begin{requirement}
\textbf{MC Guarantee-2} - The microwave shall be in setup mode in the initial
state.
\begin{verbatim}
(defs.initially_true(start)) => is_setup;
\end{verbatim}
\label{req:mic2}
\end{requirement}
%
\begin{requirement}
\textbf{MC Guarantee-3} - At the instant the microwave starts running, it shall
be in the cooking mode if the door is closed.
\begin{Verbatim}[obeytabs,fontsize=]
(defs.rising_edge(is_running) and
	       door_closed) => is_cooking;
\end{Verbatim}
\label{req:mic3}
\end{requirement}
%
\begin{requirement}
\textbf{MC Guarantee-4} - At the instant the microwave starts running, it shall
enter the suspended mode if the door is open.
\begin{Verbatim}[obeytabs,fontsize=]
(defs.rising_edge(is_running) and
	not door_closed) => is_suspended;
\end{Verbatim}
\label{req:mic4}
\end{requirement}
%
\begin{requirement}
\textbf{MC Guarantee-5} - At the instant the clear button is pressed, if the
microwave was cooking, then the microwave shall stop cooking.
\begin{Verbatim}[obeytabs,fontsize=]
(defs.rising_edge(clear) and
	is_cooking) => not is_cooking;
\end{Verbatim}
\label{req:mic5}
\end{requirement}
%
\begin{requirement}
\textbf{MC Guarantee-6} - At the instant when the clear button is pressed, if
the microwave is in suspended mode, it shall enter the setup mode.
\begin{Verbatim}[obeytabs,fontsize=]
(defs.rising_edge(clear) and
	is_suspended) => is_setup;
\end{Verbatim}
\label{req:mic6}
\end{requirement}
%
\begin{requirement}
\textbf{MC Guarantee-7} - If suspended, at the instant the start key is pressed
the microwave shall enter cooking mode if the door is closed.
\begin{Verbatim}[obeytabs]
(defs.rising_edge(start) and is_suspended
	and door_closed) => is_cooking;
\end{Verbatim}
\label{req:mic7}
\end{requirement}
%
\begin{requirement}
\textbf{MC Guarantee-8} - If seconds\_to\_cook = 0, microwave will be in setup
mode.
\begin{Verbatim}[obeytabs,fontsize=]
(seconds_to_cook = 0) => is_setup;
\end{Verbatim}
\label{req:mic8}
\end{requirement}

The contract is then translated from AADL into an equivalent Lustre program that
is then given as input to the JKind model checker~\cite{jkind}, where our
realizability algorithm is implemented as a separate feature. From JKind, the
Lustre specification is further translated into the SMT-LIB v2 format. In
reality, the program is split into two different processes that run in parallel and correspond to the
checks~\ref{eq:sbcheck} and~\ref{eq:echeck} that our synthesis algorithm is
using. Considering the fact that the
contract is realizable, we impose the negation of our target
$\forall\exists$-formula as a query to the AE-VAL skolemizer, during the last
step of \textit{ExtendCheck}. AE-VAL responds that the original formula can be
satisfied, and provides a Skolem relation, a part of which is shown in Figure~\ref{fig:skolem-rel}.

\begin{figure}
\label{fig:skolem-rel}
\begin{Verbatim}[obeytabs,fontsize=\tiny]
ite([&&
    $defs__rising_edge~1.Mode_Control_Impl_Instance__signal$0
    !($Mode_Control_Impl_Instance__seconds_to_cook$0>=0)
    !$defs__initially_true~0.Mode_Control_Impl_Instance__result$0
  ], [&&
    $Mode_Control_Impl_Instance__is_setup$0
    $defs__rising_edge~1.Mode_Control_Impl_Instance__re$0
    !$Mode_Control_Impl_Instance__is_cooking$0
    $defs__rising_edge~1.Mode_Control_Impl_Instance__signal$0
    !$_TOTAL_COMP_HIST$0
    !$_SYSTEM_ASSUMP_HIST$0
    !$Mode_Control_Impl_Instance__is_suspended$0
    !$Mode_Control_Impl_Instance__is_running$0
    !$defs__rising_edge~0.Mode_Control_Impl_Instance__re$0
    !$defs__initially_true~0.Mode_Control_Impl_Instance__b$0
    !$defs__initially_true~0.Mode_Control_Impl_Instance__result$0
    !$defs__rising_edge~2.Mode_Control_Impl_Instance__re$0
    !$defs__rising_edge~2.Mode_Control_Impl_Instance__signal$0
  ], ite([&&
      %init
      $_SYS_GUARANTEE_2$0
      !($Mode_Control_Impl_Instance__seconds_to_cook$0>=0)
      !$defs__rising_edge~1.Mode_Control_Impl_Instance__signal$0
      !$defs__initially_true~0.Mode_Control_Impl_Instance__b$0
    ], ...))
\end{Verbatim}
\caption{A portion of the Skolem relation generated for the Microwave Mode Logic}
\end{figure}
As seen above, the Skolem relation is composed of nested
\textit{if-then-else} blocks, which indicate the possible valid transitions the
implementation can follow given a specific state, without
violating the contract. Each variable appearing in the conjuncts of
the relation is named uniquely after the state that it refers to, using the
\textit{\$X} postfix, where $X$ is an integer. In addition to each state's
variables, we keep track of whether each state is initial using the variable
\textit{\%init}.

The structure of the Skolem relation above is simple enough to translate into a
mainstream program. We need implementations that are able to keep track of the
current state variables, the current inputs, as well as some history
about the variable values in previous states. This can easily be handled, for
example, in C with the use of arrays to keep record of each variable's $k$ last
values, and the use of functions that update each variable's corresponding array
to reflect the changes following a new step using the transition relation.
